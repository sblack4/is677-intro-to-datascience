\documentclass[]{article}
\usepackage{lmodern}
\usepackage{amssymb,amsmath}
\usepackage{ifxetex,ifluatex}
\usepackage{fixltx2e} % provides \textsubscript
\ifnum 0\ifxetex 1\fi\ifluatex 1\fi=0 % if pdftex
  \usepackage[T1]{fontenc}
  \usepackage[utf8]{inputenc}
\else % if luatex or xelatex
  \ifxetex
    \usepackage{mathspec}
  \else
    \usepackage{fontspec}
  \fi
  \defaultfontfeatures{Ligatures=TeX,Scale=MatchLowercase}
\fi
% use upquote if available, for straight quotes in verbatim environments
\IfFileExists{upquote.sty}{\usepackage{upquote}}{}
% use microtype if available
\IfFileExists{microtype.sty}{%
\usepackage{microtype}
\UseMicrotypeSet[protrusion]{basicmath} % disable protrusion for tt fonts
}{}
\usepackage[margin=1in]{geometry}
\usepackage{hyperref}
\hypersetup{unicode=true,
            pdftitle={Homework2},
            pdfauthor={Steven Black},
            pdfborder={0 0 0},
            breaklinks=true}
\urlstyle{same}  % don't use monospace font for urls
\usepackage{color}
\usepackage{fancyvrb}
\newcommand{\VerbBar}{|}
\newcommand{\VERB}{\Verb[commandchars=\\\{\}]}
\DefineVerbatimEnvironment{Highlighting}{Verbatim}{commandchars=\\\{\}}
% Add ',fontsize=\small' for more characters per line
\usepackage{framed}
\definecolor{shadecolor}{RGB}{248,248,248}
\newenvironment{Shaded}{\begin{snugshade}}{\end{snugshade}}
\newcommand{\KeywordTok}[1]{\textcolor[rgb]{0.13,0.29,0.53}{\textbf{#1}}}
\newcommand{\DataTypeTok}[1]{\textcolor[rgb]{0.13,0.29,0.53}{#1}}
\newcommand{\DecValTok}[1]{\textcolor[rgb]{0.00,0.00,0.81}{#1}}
\newcommand{\BaseNTok}[1]{\textcolor[rgb]{0.00,0.00,0.81}{#1}}
\newcommand{\FloatTok}[1]{\textcolor[rgb]{0.00,0.00,0.81}{#1}}
\newcommand{\ConstantTok}[1]{\textcolor[rgb]{0.00,0.00,0.00}{#1}}
\newcommand{\CharTok}[1]{\textcolor[rgb]{0.31,0.60,0.02}{#1}}
\newcommand{\SpecialCharTok}[1]{\textcolor[rgb]{0.00,0.00,0.00}{#1}}
\newcommand{\StringTok}[1]{\textcolor[rgb]{0.31,0.60,0.02}{#1}}
\newcommand{\VerbatimStringTok}[1]{\textcolor[rgb]{0.31,0.60,0.02}{#1}}
\newcommand{\SpecialStringTok}[1]{\textcolor[rgb]{0.31,0.60,0.02}{#1}}
\newcommand{\ImportTok}[1]{#1}
\newcommand{\CommentTok}[1]{\textcolor[rgb]{0.56,0.35,0.01}{\textit{#1}}}
\newcommand{\DocumentationTok}[1]{\textcolor[rgb]{0.56,0.35,0.01}{\textbf{\textit{#1}}}}
\newcommand{\AnnotationTok}[1]{\textcolor[rgb]{0.56,0.35,0.01}{\textbf{\textit{#1}}}}
\newcommand{\CommentVarTok}[1]{\textcolor[rgb]{0.56,0.35,0.01}{\textbf{\textit{#1}}}}
\newcommand{\OtherTok}[1]{\textcolor[rgb]{0.56,0.35,0.01}{#1}}
\newcommand{\FunctionTok}[1]{\textcolor[rgb]{0.00,0.00,0.00}{#1}}
\newcommand{\VariableTok}[1]{\textcolor[rgb]{0.00,0.00,0.00}{#1}}
\newcommand{\ControlFlowTok}[1]{\textcolor[rgb]{0.13,0.29,0.53}{\textbf{#1}}}
\newcommand{\OperatorTok}[1]{\textcolor[rgb]{0.81,0.36,0.00}{\textbf{#1}}}
\newcommand{\BuiltInTok}[1]{#1}
\newcommand{\ExtensionTok}[1]{#1}
\newcommand{\PreprocessorTok}[1]{\textcolor[rgb]{0.56,0.35,0.01}{\textit{#1}}}
\newcommand{\AttributeTok}[1]{\textcolor[rgb]{0.77,0.63,0.00}{#1}}
\newcommand{\RegionMarkerTok}[1]{#1}
\newcommand{\InformationTok}[1]{\textcolor[rgb]{0.56,0.35,0.01}{\textbf{\textit{#1}}}}
\newcommand{\WarningTok}[1]{\textcolor[rgb]{0.56,0.35,0.01}{\textbf{\textit{#1}}}}
\newcommand{\AlertTok}[1]{\textcolor[rgb]{0.94,0.16,0.16}{#1}}
\newcommand{\ErrorTok}[1]{\textcolor[rgb]{0.64,0.00,0.00}{\textbf{#1}}}
\newcommand{\NormalTok}[1]{#1}
\usepackage{graphicx,grffile}
\makeatletter
\def\maxwidth{\ifdim\Gin@nat@width>\linewidth\linewidth\else\Gin@nat@width\fi}
\def\maxheight{\ifdim\Gin@nat@height>\textheight\textheight\else\Gin@nat@height\fi}
\makeatother
% Scale images if necessary, so that they will not overflow the page
% margins by default, and it is still possible to overwrite the defaults
% using explicit options in \includegraphics[width, height, ...]{}
\setkeys{Gin}{width=\maxwidth,height=\maxheight,keepaspectratio}
\IfFileExists{parskip.sty}{%
\usepackage{parskip}
}{% else
\setlength{\parindent}{0pt}
\setlength{\parskip}{6pt plus 2pt minus 1pt}
}
\setlength{\emergencystretch}{3em}  % prevent overfull lines
\providecommand{\tightlist}{%
  \setlength{\itemsep}{0pt}\setlength{\parskip}{0pt}}
\setcounter{secnumdepth}{0}
% Redefines (sub)paragraphs to behave more like sections
\ifx\paragraph\undefined\else
\let\oldparagraph\paragraph
\renewcommand{\paragraph}[1]{\oldparagraph{#1}\mbox{}}
\fi
\ifx\subparagraph\undefined\else
\let\oldsubparagraph\subparagraph
\renewcommand{\subparagraph}[1]{\oldsubparagraph{#1}\mbox{}}
\fi

%%% Use protect on footnotes to avoid problems with footnotes in titles
\let\rmarkdownfootnote\footnote%
\def\footnote{\protect\rmarkdownfootnote}

%%% Change title format to be more compact
\usepackage{titling}

% Create subtitle command for use in maketitle
\newcommand{\subtitle}[1]{
  \posttitle{
    \begin{center}\large#1\end{center}
    }
}

\setlength{\droptitle}{-2em}

  \title{Homework2}
    \pretitle{\vspace{\droptitle}\centering\huge}
  \posttitle{\par}
    \author{Steven Black}
    \preauthor{\centering\large\emph}
  \postauthor{\par}
      \predate{\centering\large\emph}
  \postdate{\par}
    \date{2/12/2019}


\begin{document}
\maketitle

\subsection{Homework 2 for IS677 Intro to Data
Science}\label{homework-2-for-is677-intro-to-data-science}

\begin{enumerate}
\def\labelenumi{\arabic{enumi}.}
\tightlist
\item
  The household data is part of a data set collected from a survey of
  household expenditure and give the expenditure of 20 single men and 20
  single women on four commodity groups. The units of expenditure are
  Hong Kong dollars, and the four commodity groups are as follows:
\end{enumerate}

\begin{itemize}
\tightlist
\item
  housing: housing, including fuel and light,
\item
  food: foodstuff, including alcohol and tobacco,
\item
  goods: other goods, including clothing, footwear and durable goods,
\item
  service: services, including transport and vehicles.
\end{itemize}

The aim of the survey was to investigate how the division of household
expenditure between the four commodity groups depends on total
expenditure and to find out whether this relationship differs for men
and women. Load the data into a data frame in R and use appropriate
graphical methods to answer these questions and state your conclusions.

\begin{Shaded}
\begin{Highlighting}[]
\NormalTok{household <-}\StringTok{ }\KeywordTok{read.table}\NormalTok{(}\StringTok{"household.txt"}\NormalTok{, }\DataTypeTok{header =} \OtherTok{TRUE}\NormalTok{, }\DataTypeTok{sep =} \StringTok{","}\NormalTok{)}
\end{Highlighting}
\end{Shaded}

\begin{Shaded}
\begin{Highlighting}[]
\NormalTok{group_sums <-}\StringTok{ }\KeywordTok{colSums}\NormalTok{(household[,}\DecValTok{1}\OperatorTok{:}\DecValTok{4}\NormalTok{])}
\NormalTok{pct <-}\StringTok{ }\KeywordTok{round}\NormalTok{(group_sums}\OperatorTok{/}\KeywordTok{sum}\NormalTok{(group_sums)}\OperatorTok{*}\DecValTok{100}\NormalTok{)}
\NormalTok{lbls <-}\StringTok{ }\KeywordTok{paste}\NormalTok{(}\KeywordTok{labels}\NormalTok{(group_sums), pct, }\StringTok{"%"}\NormalTok{)}
\KeywordTok{pie}\NormalTok{(group_sums,}
    \DataTypeTok{main =} \StringTok{"Household Expenditure: }\CharTok{\textbackslash{}n}\StringTok{ Across 4 Commodity Groups for Men & Women"}\NormalTok{,}
    \DataTypeTok{labels =}\NormalTok{ lbls)}
\end{Highlighting}
\end{Shaded}

\includegraphics{Homework2_files/figure-latex/unnamed-chunk-2-1.pdf}

\begin{Shaded}
\begin{Highlighting}[]
\NormalTok{groups_sums_grouped <-}\StringTok{ }\KeywordTok{aggregate}\NormalTok{(. }\OperatorTok{~}\StringTok{ }\NormalTok{household}\OperatorTok{$}\NormalTok{gender, household, sum)}
\NormalTok{group_sums_men <-}\StringTok{ }\KeywordTok{as.numeric}\NormalTok{(groups_sums_grouped[}\DecValTok{2}\NormalTok{, }\DecValTok{2}\OperatorTok{:}\DecValTok{5}\NormalTok{])}
\NormalTok{pct_men <-}\StringTok{ }\KeywordTok{round}\NormalTok{(group_sums_men}\OperatorTok{/}\KeywordTok{sum}\NormalTok{(group_sums_men)}\OperatorTok{*}\DecValTok{100}\NormalTok{)}
\NormalTok{lbls_men <-}\StringTok{ }\KeywordTok{paste}\NormalTok{(}\KeywordTok{labels}\NormalTok{(group_sums), pct_men, }\StringTok{"%"}\NormalTok{)}
\KeywordTok{pie}\NormalTok{(group_sums_men,}
    \DataTypeTok{main =} \StringTok{"Household Expenditure: }\CharTok{\textbackslash{}n}\StringTok{ Across 4 Commodity Groups for Men"}\NormalTok{,}
    \DataTypeTok{labels =}\NormalTok{ lbls_men)}
\end{Highlighting}
\end{Shaded}

\includegraphics{Homework2_files/figure-latex/unnamed-chunk-3-1.pdf}

\begin{Shaded}
\begin{Highlighting}[]
\NormalTok{group_sums_women <-}\StringTok{ }\KeywordTok{as.numeric}\NormalTok{(groups_sums_grouped[}\DecValTok{1}\NormalTok{, }\DecValTok{2}\OperatorTok{:}\DecValTok{5}\NormalTok{])}
\NormalTok{pct_women <-}\StringTok{ }\KeywordTok{round}\NormalTok{(group_sums_women}\OperatorTok{/}\KeywordTok{sum}\NormalTok{(group_sums_women)}\OperatorTok{*}\DecValTok{100}\NormalTok{)}
\NormalTok{lbls_women <-}\StringTok{ }\KeywordTok{paste}\NormalTok{(}\KeywordTok{labels}\NormalTok{(group_sums), pct_women, }\StringTok{"%"}\NormalTok{)}
\KeywordTok{pie}\NormalTok{(group_sums_women,}
    \DataTypeTok{main =} \StringTok{"Household Expenditure: }\CharTok{\textbackslash{}n}\StringTok{ Across 4 Commodity Groups for Women"}\NormalTok{,}
    \DataTypeTok{labels =}\NormalTok{ lbls_women)}
\end{Highlighting}
\end{Shaded}

\includegraphics{Homework2_files/figure-latex/unnamed-chunk-4-1.pdf}

\begin{Shaded}
\begin{Highlighting}[]
\KeywordTok{library}\NormalTok{(vcd)}
\end{Highlighting}
\end{Shaded}

\begin{verbatim}
## Loading required package: grid
\end{verbatim}

\begin{Shaded}
\begin{Highlighting}[]
\NormalTok{gsg <-}\StringTok{ }\NormalTok{groups_sums_grouped[}\DecValTok{2}\OperatorTok{:}\DecValTok{5}\NormalTok{]}
\KeywordTok{row.names}\NormalTok{(gsg) <-}\StringTok{ }\KeywordTok{c}\NormalTok{(}\StringTok{"Women"}\NormalTok{, }\StringTok{"Men"}\NormalTok{)}
\KeywordTok{spine}\NormalTok{(}\KeywordTok{as.matrix}\NormalTok{(gsg), }\DataTypeTok{main =} \StringTok{"Spinogram of Expenditures }\CharTok{\textbackslash{}n}\StringTok{ Commodity Group Spending between Women and Men"}\NormalTok{)}
\end{Highlighting}
\end{Shaded}

\includegraphics{Homework2_files/figure-latex/unnamed-chunk-5-1.pdf} The
pie charts and the spinogram above clearly show there is a difference in
spending habits between men and woman. Men spend roughly the same
proportion of money on each of the four categories while women spend
much more on housing and goods. It's possible that women like remodeling
their kitchens and lots of clothes and so spend more money in the
housing and goods categories. Another explanation is that men eat a lot
and are lazy so they spend more in the food and service categories.
Explanations of this type rarely come down to one factor and so it is
likely that the true cause is a little of both and possibly other
factors.

\begin{enumerate}
\def\labelenumi{\arabic{enumi}.}
\setcounter{enumi}{1}
\tightlist
\item
  Mortality rates per 100,000 from male suicides for a number of age
  groups and a number of countries are provided in the suicides2 data
  set. Load the data into R and construct side-by-side box plots for the
  data from different age groups. Comment on what the graphic tells us
  about the data.
\end{enumerate}

\begin{Shaded}
\begin{Highlighting}[]
\NormalTok{suicides <-}\StringTok{ }\KeywordTok{read.table}\NormalTok{(}\StringTok{"suicides.txt"}\NormalTok{, }\DataTypeTok{header =} \OtherTok{TRUE}\NormalTok{, }\DataTypeTok{sep =} \StringTok{","}\NormalTok{)}
\end{Highlighting}
\end{Shaded}

\begin{Shaded}
\begin{Highlighting}[]
\NormalTok{num_rows <-}\StringTok{ }\KeywordTok{nrow}\NormalTok{(suicides)}
\KeywordTok{boxplot}\NormalTok{(suicides,}
        \DataTypeTok{main =} \KeywordTok{paste}\NormalTok{(}\StringTok{"Boxplots for 100k Male Suicides for"}\NormalTok{, }\KeywordTok{toString}\NormalTok{(num_rows), }\StringTok{"Countries"}\NormalTok{),}
        \DataTypeTok{xlab =} \StringTok{"Age Groups"}\NormalTok{,}
        \DataTypeTok{ylab =} \StringTok{"Number of Suicides"}\NormalTok{,}
        \DataTypeTok{names =} \KeywordTok{c}\NormalTok{(}\StringTok{"25 - 34"}\NormalTok{, }\StringTok{"35 - 44"}\NormalTok{, }\StringTok{"45 - 54"}\NormalTok{, }\StringTok{"55 - 64"}\NormalTok{, }\StringTok{"65 - 74"}\NormalTok{))}
\end{Highlighting}
\end{Shaded}

\includegraphics{Homework2_files/figure-latex/unnamed-chunk-7-1.pdf}
Oddly we see more suicides the older men get. There should be more
younger men, by the logic that older men must have been younger men at
some point, so if the rate of suicides across age groups is constant
then there should be a greater number of suicides for younger men. Also
strange is the greater variance in suicides between in the groups 25 -
34 and 55 - 64. These age groups roughly correlate to the quarter-life
and mid-life chrisis. An explanation for this phenomenon could be that
countries with greater pressures for men to succeed see high rates of
suicides at these junctures while countries that have less pressure see
normal amounts of suicide.


\end{document}
